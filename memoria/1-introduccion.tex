\chapter{Introducción}\label{cap1}


Tradicionalmente, las empresas han dependido de métodos estadísticos convencionales para realizar previsiones de demanda. Sin embargo, estos enfoques a menudo no capturan la complejidad y los patrones no lineales presentes en los datos de ventas históricos, lo que puede llevar a predicciones inexactas. En los últimos años, los avances en el campo del aprendizaje automático han abierto nuevas posibilidades para mejorar la precisión de las previsiones, permitiendo a las empresas obtener insights más profundos y ajustados a la realidad del mercado.

El presente trabajo tiene como objetivo principal explorar la aplicación de algoritmos de aprendizaje automático en la previsión de demanda, utilizando los datos de ventas de una empresa de recambios de coche como caso de estudio. Para ello, se empleará la librería Darts en Python, una herramienta especializada en forecasting, que ofrece una variedad de modelos y técnicas avanzadas para la predicción de series temporales. A través de la implementación y evaluación de diferentes modelos, se busca identificar aquellos que proporcionen las predicciones más precisas y robustas, con el fin de mejorar la capacidad de la empresa para anticipar la demanda de sus productos.

El trabajo está estructurado de la siguiente manera: en primer lugar, se realiza una revisión de la literatura existente en el campo de la previsión de demanda y el aprendizaje automático, con un enfoque especial en las aplicaciones en la industria de recambios de coche. A continuación, se describe la metodología utilizada, incluyendo la descripción de los datos, el preprocesamiento y la implementación de los modelos. Posteriormente, se presentan y analizan los resultados obtenidos, comparando el rendimiento de los diferentes algoritmos. Finalmente, se discuten las conclusiones del estudio y se sugieren posibles líneas de investigación futura.


\section{Uno}



BLOQUE A1: NIVEL TECNOLÓGICO

BLOQUE A2: PROBLEMA DE LA INDUSTRIA A RESOLVER

Relevando de la acción de repuestos en actividad a soporte en servicios

Caso de uso

BLOQUE B: DESARROLLO DE LA SOLUCIÓN

BLOQUE C: COMO LA SOLUCIÓN HA SOLUCIONADO EL PROBLEMA Y LÍNEAS DE TRABAJO FUTURA








La previsión o \textit{forecast}, definido según  es la predicción de un evento o eventos futuros. En un mercado global cada vez más competitivo, el empleo de técnicas de previsión de la demanda más precisas cobran un papel indispensable a la hora de planificar las actividades productivas dentro de la empresa, y, por tanto, los recursos que se requieren para satisfacer dicha demanda. Según el análisis a emplear, se puede diferenciar dos formas de previsión: previsión subjetiva, métodos intuitivos que pueden no estar relacionados con los datos pasados y previsión cuantitativa, la cual supone que las características de tendencias de los datos del pasado continuarán en el futuro  A lo largo del documento, trabajaremos con esta última mediante un enfoque de modelo de regresión. Un análisis de regresión estudia las relaciones entre diferentes variables, cuantificando la respuesta de una variable ante la variación de otras  De este modo, con el objetivo de buscar una mayor precisión en la estimación de la demanda, el proyecto estudiará este enfoque de regresión mediante modelos multivariables, estos son, modelos que empleen múltiples variables de entrada analizadas a lo largo del tiempo para obtener como salida el valor de predicción. La figura 2.1 recoge el esquema multivariable en el sistema de previsión.