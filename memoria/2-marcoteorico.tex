\chapter{Marco teórico}\label{cap2}


%La previsión o \textit{forecast}, definido según  es la predicción de un evento o eventos futuros. En un mercado global cada vez más competitivo, el empleo de técnicas de previsión de la demanda más precisas cobran un papel indispensable a la hora de planificar las actividades productivas dentro de la empresa, y, por tanto, los recursos que se requieren para satisfacer dicha demanda. Según el análisis a emplear, se puede diferenciar dos formas de previsión: previsión subjetiva, métodos intuitivos que pueden no estar relacionados con los datos pasados y previsión cuantitativa, la cual supone que las características de tendencias de los datos del pasado continuarán en el futuro  A lo largo del documento, trabajaremos con esta última mediante un enfoque de modelo de regresión. Un análisis de regresión estudia las relaciones entre diferentes variables, cuantificando la respuesta de una variable ante la variación de otras  De este modo, con el objetivo de buscar una mayor precisión en la estimación de la demanda, el proyecto estudiará este enfoque de regresión mediante modelos multivariables, estos son, modelos que empleen múltiples variables de entrada analizadas a lo largo del tiempo para obtener como salida el valor de predicción. La figura 2.1 recoge el esquema multivariable en el sistema de previsión.


La previsión de demanda ha sido un área de investigación activa durante varias décadas debido a su impacto directo en la eficiencia operativa y la rentabilidad de las empresas. Desde la industria minorista hasta la manufactura, la capacidad de predecir la demanda de productos de manera precisa es crucial para optimizar la cadena de suministro, evitar el exceso de inventario o las roturas de stock, y mejorar la satisfacción del cliente. En la industria de recambios de coche, en particular, la variabilidad en la demanda de productos específicos añade una capa de complejidad a la previsión, ya que está influenciada por factores como la estacionalidad, las tendencias tecnológicas y los hábitos de mantenimiento de los consumidores.

