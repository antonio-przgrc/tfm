\chapter{Introducción}\label{cap1}

La predicción de ventas es un factor importante para las empresas, ya que permite adelantarse al mercado y planificar en consecuencia. Tradicionalmente, se ha utilizado métodos estadísticos convencionales para realizar las previsiones de la demanda. Sin embargo, estos enfoques no capturan a menudo la complejidad y los factores no lineales que aparecen en los datos de ventas históricos. Con el avance en el campo del aprendizaje automático se ha abierto todo un abanico de opciones para el desarrollo de modelos complejos que, potencialmente, son capaces de proveer resultados de mayor calidad y ajustarse a la realidad del mercado de las empresas.

El presente trabajo tiene el objetivo de explorar la aplicación de diversos algoritmos de aprendizaje automático sobre el histórico de ventas de una empresa distribuidora de recambios de coche como caso de estudio y evaluar los resultados obtenidos. Para ello, se utiliza Python \cite*{python} como lenguaje de programación y la librería Darts \cite*{darts}, una herramienta especializada en \textit{forecasting} con series temporales. Cuenta con distintos modelos tanto estadísticos como de aprendizaje automático, además de distintas utilidades para el entrenamiento y la evaluación de los modelos con el fin de mejorar las predicciones.

La estructura del trabajo comienza con una revisión de la literatura existente en el campo de la previsión de demanda y el aprendizaje automático, dando contexto de las técnicas actuales usadas en la empresa de recambios de la que se utilizan los datos. A continuación, se describe la metodología utilizada, incluyendo una descripción de los datos, el preprocesamiento y la implementación de los distintos modelos. Posteriormente, se presentan y analizan los resultados obtenidos por los distintos algoritmos. Por último, se discuten las conclusiones del estudio y se sugieren futuras líneas de investigación.


% BLOQUE A1: NIVEL TECNOLÓGICO

% BLOQUE A2: PROBLEMA DE LA INDUSTRIA A RESOLVER

% Relevando de la acción de repuestos en actividad a soporte en servicios

% Caso de uso<<<<<<<<<<<<<<<<<<<ss

% BLOQUE B: DESARROLLO DE LA SOLUCIÓN

% BLOQUE C: COMO LA SOLUCIÓN HA SOLUCIONADO EL PROBLEMA Y LÍNEAS DE TRABAJO FUTURA