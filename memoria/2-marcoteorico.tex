\chapter{Marco teórico}\label{cap2}

El \textit{forecasting} es la predicción uno o varios eventos futuros \cite{intro}. Esta simple descripción permite que un ámbito de investigación tenga impacto en diversos sectores como los mercados, las finanzas o la meteorología. Tanto la predicción del valor de una acción en los próximos minutos en bolsa como la predicción de las ventas de una tienda durante el próximo trimestre se benefician de los avances en este ámbito. Las predicciones pueden ser a corto plazo, medio y largo plazo, por lo que pueden ser utilizadas para optimizar las decisiones operativas, tácticas y estratégicas de las empresas. En el contexto empresarial, una buena previsión permite anticiparse a cambios en la demanda ajustando la producción, el suministro y la distribución para maximizar la rentabilidad y minimizar los riesgos asociados con el exceso o falta de stock. 

\section{Métodos tradicionales}

\section{Aprendizaje automático}


\section{Modelos}

Los modelos matemáticos son herramientas matemáticas o estadísticas diseñadas para identificar patrones en las series de datos con el fin de predecir valores futuros. Cada modelo tiene una arquitectura que captura diferentes características y distintos tipos de patrones de los datos, como la tendencia, la estacionalidad y las autocorrelaciones, utilizando dicha información para generar predicciones basadas en el comportamiento pasado de la serie temporal.

\subsection{Prophet}

Prophet \cite{prophet} es una herramienta desarrollada por Facebook (ahora conocida como Meta) para el análisis de series temporales. Está especializada en datos que presentan patrones estacionales y tendencia. Está orientado al usuario final sin requerir configuración compleja específica para su aplicación. Se utiliza como librería en Python o R \cite{R}. El modelo incluido descompone la serie temporal en tendencia y estacionalidad. Además, permite añadir los festivos locales para mejorar las predicciones.

Se utiliza un modelo en el que la tendencia se modela de forma flexible, permitiendo cambios no lineales mediante \textit{changepoints}. Evita los valores atípicos y los datos faltantes, dando robustez al modelo frente a series temporales ruidosas o incompletas.

Su uso en este trabajo es de \textit{plug and play}, ya que se instancia la librería por defecto, a excepción de la especificación de los festivos de España (la librería no incluye la posiblidad de añadir los festivos de Andalucía).

Es necesario destacar que es el único modelo de los implementados que no depende de PyTorch \cite{pytorch}.

\subsection{RNN}
\subsection{N-BEATS y N-HiTS}
\subsection{TCN}
\subsection{DLinear y NLinear}
\subsection{TiDE}
\subsection{TSMixer}