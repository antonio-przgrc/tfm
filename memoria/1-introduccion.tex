\chapter{Introducción}\label{cap1}

Hola hola en este \ref{cap1}

\section{Uno}

BLOQUE A1: NIVEL TECNOLÓGICO

BLOQUE A2: PROBLEMA DE LA INDUSTRIA A RESOLVER

Relevando de la acción de repuestos en actividad a soporte en servicios

Caso de uso

BLOQUE B: DESARROLLO DE LA SOLUCIÓN

BLOQUE C: COMO LA SOLUCIÓN HA SOLUCIONADO EL PROBLEMA Y LÍNEAS DE TRABAJO FUTURA








La previsión o \textit{forecast}, definido según \cite{douglas}, es la predicción de un evento o eventos futuros. En un mercado global cada vez más competitivo, el empleo de técnicas de previsión de la demanda más precisas cobran un papel indispensable a la hora de planificar las actividades productivas dentro de la empresa, y, por tanto, los recursos que se requieren para satisfacer dicha demanda. Según el análisis a emplear, se puede diferenciar dos formas de previsión: previsión subjetiva, métodos intuitivos que pueden no estar relacionados con los datos pasados y previsión cuantitativa, la cual supone que las características de tendencias de los datos del pasado continuarán en el futuro \cite{bovas}. A lo largo del documento, trabajaremos con esta última mediante un enfoque de modelo de regresión. Un análisis de regresión estudia las relaciones entre diferentes variables, cuantificando la respuesta de una variable ante la variación de otras \cite{bovas}. De este modo, con el objetivo de buscar una mayor precisión en la estimación de la demanda, el proyecto estudiará este enfoque de regresión mediante modelos multivariables, estos son, modelos que empleen múltiples variables de entrada analizadas a lo largo del tiempo para obtener como salida el valor de predicción. La figura 2.1 recoge el esquema multivariable en el sistema de previsión.