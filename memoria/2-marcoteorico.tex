\chapter{Marco teórico}\label{cap2}

El \textit{forecasting} es la predicción uno o varios eventos futuros \cite{intro}. Esta simple descripción permite que un ámbito de investigación tenga impacto en diversos sectores como los mercados, las finanzas o la meteorología. Tanto la predicción del valor de una acción en los próximos minutos en bolsa como la predicción de las ventas de una tienda durante el próximo trimestre se benefician de los avances en este ámbito. Las predicciones pueden ser a corto plazo, medio y largo plazo, por lo que pueden ser utilizadas para optimizar las decisiones operativas, tácticas y estratégicas de las empresas. En el contexto empresarial, una buena previsión permite anticiparse a cambios en la demanda ajustando la producción, el suministro y la distribución para maximizar la rentabilidad y minimizar los riesgos asociados con el exceso o falta de stock. 

\section{Métodos tradicionales}

\section{Aprendizaje automático}
