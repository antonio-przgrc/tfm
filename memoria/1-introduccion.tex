\chapter{Introducción}\label{cap1}

La previsión de demanda es una herramienta importante para la planificación estratégica de empresas de múltiples sectores como el comercio minorista, la producción industrial y la logística. En el sector del \textit{aftermarket} del automóvil, la capacidad de predecir con la suficiente precisión la demanada de productos es fundamental para optimizar la cadena de suministro, reducir los costes de inventario y mejorar el servicio al cliente. La variabilidad de la demanda debida a factores como la estacionalidad, las tendencias económicas, las normativas medioambientales y los cambios en el comportamiento de los consumidores hace que las empresas se enfrenten a desafíos considerables a la hora de planificar su cadena de suministros de forma eficiente.

Las predicciones se pueden realizar mediante métodos intuitivos, en basados en el \textit{expertise} o experiencia y trayectoria profesional de los directores de operaciones de la empresa, o mediante métodos cuantitativos, utilizando modelos matemáticos entrenados con datos históricos. Tradicionalmente, se han utilizado métodos estadísticos convencionales para realizar predicciones, como pueden ser modelos de media móvil, suavizado exponencial o el método de Holt-Winters \cite{winters}. Sin embargo, los avances en el campo de la inteligencia artificial han permitido el desarrollo de modelos de aprendizaje automático que son capaces de interpretar patrones complejos en el histórico de datos. Además, estos algoritmos pueden integrar variables externas que impacten en los datos de estudio, como puede ser el precio del combustible o la meteorología.

En este contexto, se realiza un caso de estudio para una empresa distribuidora de recambios de automóviles situada en Andalucía, con puntos de venta y distribución ubicados en zonas rurales de las provincias de Cádiz y Málaga. Esta empresa cuenta con un volumen de ventas creciente y su intención es realizar un estudio de las ventas de sus productos para realizar previsiones a futuro. El objetivo es comparar distintos modelos de predicción y ver cuál se ajusta más a las series de datos de la empresa y, con ello, poder realizar mejores propuestas a proveedores. 

Para ello, se utiliza el lenguaje de programación Python \cite{python}, uno de los más utilizados en el campo de la ciencia de datos, junto con la librería Darts \cite{darts}, especializada en \textit{forecasting} en series temporales.

\section{Estructura del trabajo}

El trabajo comienza en el Capítulo \ref{cap2} donde se realiza una revisión del marco teórico, se realiza una revisión de los algoritmos de predicción de series temporales más relevantes para el caso de estudio y una descripción general del ámbito del \textit{forecast}.

A continuación, en el Capítulo \ref{cap3}: Metodología, se detallan las series históricas de datos utilizados, los algoritmos implementados y así como las métricas para la evaluación de los resultados.

En el Capítulo \ref{cap4}: Desarrollo, se describe la implementación y el código utilizado para el procesamiento de los datos, el entrenamiento de los modelos y las predicciones realizadas.

Una vez realizadas las predicciones, se estudian los resultados obtenidos incluyendo las gráficas y métricas en el Capítulo \ref{cap5}: Resultados.

Para finalizar el trabajo, se incluye el Capítulo \ref{cap6}: Conclusiones, donde se realizan las reflexiones finales del estudio se sugieren propuestas para proyectos futuros.


% BLOQUE A1: NIVEL TECNOLÓGICO

% BLOQUE A2: PROBLEMA DE LA INDUSTRIA A RESOLVER

% Relevando de la acción de repuestos en actividad a soporte en servicios

% Caso de uso<<<<<<<<<<<<<<<<<<<ss

% BLOQUE B: DESARROLLO DE LA SOLUCIÓN

% BLOQUE C: COMO LA SOLUCIÓN HA SOLUCIONADO EL PROBLEMA Y LÍNEAS DE TRABAJO FUTURA